\subsection{Connection to Hyperelliptic Curve Discrete Logarithm Problem (HCDLP)}\label{subsec:hcdlp-connection}
            This section establishes a profound, albeit computationally expensive, connection between the Affine Iterated Inversion Problem (AIIP) and the discrete logarithm problem in a specific family of high-genus hyperelliptic curves. While this does not constitute a polynomial-time reduction suitable for a direct security proof, it provides strong number-theoretic evidence for the hardness of AIIP by embedding it into a well-studied mathematical framework believed to be resistant to quantum attacks.
            \begin{theorem}[Connection between AIIP and HCDLP]\label{thm:aiip-hcdlp-connection}
                Let \( f(x) = x^2 + \alpha \) where \( \alpha \in \mathbb{F}_q^* \) is a quadratic non-residue. The problem of solving AIIP for \( (f, n, y) \) can be embedded into the problem of computing a discrete logarithm in the Jacobian of a hyperelliptic curve \( C_{n,y} \) of genus \( g_n = 2^{n-1} - 1 \) over \( \mathbb{F}_q \). This embedding demonstrates that any algorithm solving DLP in such high-genus Jacobians would also solve AIIP. The process of constructing this embedding, however, has complexity exponential in \( n \) and thus does not constitute a polynomial-time reduction.
            \end{theorem}
            \begin{proof}
                The connection is constructed explicitly through the following steps:
                \textbf{Step 1: Curve Construction.}
                    Given an AIIP instance \( (f_\alpha, n, y) \), we define the hyperelliptic curve \( C_{n,y} \) by its affine model:
                    \begin{equation}\label{eq:curve-connection}
                        C_{n,y}: v^2 = F_n(u) - y,
                    \end{equation}
                    where \( F_n(u) = \iter{n}(u) \) is the explicit polynomial representing the \( n \)-th iterate of \( f_\alpha \). The projective closure of this curve is non-singular for generic \( y \) due to the critical non-recurrence of \( f_\alpha \) (a consequence of \( \alpha \) being a non-residue), which ensures \( F_n(u) - y \) is square-free \cite[Chapter 4]{Silverman2007}.
                \textbf{Step 2: Genus Calculation.}
                    The polynomial \( F_n(u) \) has degree \( 2^n \). For a hyperelliptic curve defined by \( v^2 = h(u) \) with \( \deg(h) = 2^n \) and \( h(u) \) square-free, the genus is given by
                    \begin{equation}
                        g_n = \lfloor (2^n - 1)/2 \rfloor = 2^{n-1} - 1,
                    \end{equation}
                    as per \cite[Proposition 7.4.24]{Cohen2010}.
                \textbf{Step 3: Embedding the AIIP Solution.}
                    Let \( x_0 \in \mathbb{F}_q \) be a solution to the AIIP instance, i.e., \( F_n(x_0) = y \). This implies that the point \( P_0 = (x_0, 0) \) is a rational point on the curve \( C_{n,y} \). We can then construct a degree-zero divisor on the Jacobian \( \jacobian(C_{n,y}) \):
                    \begin{equation}
                        D_{x_0} = [P_0] - [\infty].
                    \end{equation}
                    The divisor \( D_{x_0} \) is a element of finite order in \( \jacobian(C_{n,y}) \).
                \textbf{Step 4: Link to DLP.}
                    Let \( D_1 \) be a publicly known divisor of large prime order \( \ell \) in \( \jacobian(C_{n,y}) \). If an adversary could solve the discrete logarithm problem in this Jacobian, they could find an integer \( m \) (mod \( \ell \)) such that:
                    \begin{equation}
                        D_{x_0} \sim m D_1,
                    \end{equation}
                    where \( \sim \) denotes linear equivalence. Knowledge of this discrete logarithm \( m \) could, with high probability, be used to recover the reduced divisor representation of \( D_{x_0} \), which would include the point \( P_0 = (x_0, 0) \) in its support, thereby revealing the solution \( x_0 \) \cite[Section 14.2]{Cohen2010}.
                \textbf{Complexity of the Embedding.}
                    The critical observation is that this embedding is computationally expensive. Symbolically computing the polynomial \( F_n(u) = \iter{n}(u) \) requires \( O(n) \) polynomial compositions, resulting in a polynomial of degree \( 2^n \) with \( O(2^n) \) coefficients. This process has time and space complexity \( \Omega(2^n) \), which is exponential in the security parameter \( n \). Furthermore, subsequent operations, such as finding a base divisor \( D_1 \) of prime order in a Jacobian of size \( \approx q^{g_n} = q^{2^{n-1}-1} \), are also intractable for cryptographically relevant values of \( n \). Therefore, this construction provides a mathematical \emph{connection} but not an efficient \emph{reduction}.
            \end{proof}
            \begin{remark}\label{rm:complexity-remark}
                The exponential complexity of this embedding lies in the \emph{construction} of the curve equation. This is an inherent property of the iterated map \( f^{(n)} \), whose degree grows exponentially. This does not negate the value of the connection; it simply means that the security of AIIP cannot be \emph{based} on the hardness of HCDLP via a standard polynomial-time reduction. Instead, it demonstrates that AIIP inherits a similar \emph{type} of number-theoretic hardness, which is valuable evidence for its post-quantum potential.
            \end{remark}
            \begin{remark}\label{rm:quantum-resistance}
                The significance of this connection is amplified by the quantum resistance profile of the problems involved. Shor's algorithm \cite{Shor1997} solves the DLP in elliptic curves and finite fields in polynomial time but does not apply to the DLP in high-genus Jacobians (\( g \geq 2 \)) \cite{Childs2014}. The best known quantum algorithms for high-genus HCDLP, such as those based on Childs et al.'s index calculus, remain subexponential but still superpolynomial \cite{Childs2014}. This provides a strong argument that AIIP, by virtue of its connection to HCDLP, resides in a complexity class that is currently believed to be resistant to quantum cryptanalysis.
            \end{remark}
            \subsubsection{Implications and Cryptographic Interpretation}
                The connection formalized in Theorem~\ref{thm:aiip-hcdlp-connection} positions AIIP within a deep algebraic geometry framework. It demonstrates that inverting the iterated map \( f^{(n)} \) is \emph{at least as hard} as solving the DLP on a specifically structured hyperelliptic curve of exponential genus. While we cannot use this to derive a concrete security parameter mapping via reductionist arguments, it provides compelling auxiliary evidence for the problem's hardness.
                This dual foundation, combinatorial through the MQ reduction (Section~\ref{subsec:mq-reduction}) and number-theoretic through the HCDLP connection, makes AIIP a uniquely well-motivated problem for post-quantum cryptography. An adversary would need to break both the structured MQ system \emph{and} overcome the number-theoretic hardness of the associated curve to break AIIP in general.
        
